%!TEX root = Main.tex
\documentclass[Main]{subfiles}

\begin{document}

\section{Theory} % (fold)
\label{sec:theory}

	In this section a brief overview of the theory of compressive sensing and RACS is given, to provide a basic understanding of the subject and to remove ambiguities in terminology and notation.
	For a thorough review of CS theory the read is referred to other literature \cite{Fazel2011,Candes2008}.

	\subsection{Sparsity and Basis} % (fold)
	\label{sub:sparsity_and_basis}

		It is well known that natural signals are often sparse in some domain (time, frequency etc.)\cite{Candes2008}.
		They are however often sampled at a high resolution in that, or indeed another, domain, to accurately present information in the denser parts of the signal.
		For this reason, lots of data with little to no extra information is being collected and maybe also transmitted, causing an overhead at the expense of for example the lifetime of a Wireless Sensor Network (WSN).

		The theory of CS states that it is possible to do a sparse sampling of a dense signal, sparse in a different domain, and, with high accuracy, reconstruct the sparse representation, and hence the original signal.

		\subsubsection{Condensed representation} % (fold)
		\label{sub:condensed_representation}

			When a signal is sparse in the domain it is samples in, e.g. a high sample-rate time-series sampling infrequently occurring events, one can make a \emph{condensed representation} of this signal by simple dimensionality reduction.

			\begin{equation}
				\mathbf{y} = \Phi \mathbf{x}
				\label{eq:dimReduc}
			\end{equation}

			In (\ref{eq:dimReduc}) $\mathbf{x}$ is an $N$-dimensional $K$-sparse vector of the original signal, $\Phi$ is an $ M \times N $ dimensionality reduction matrix and $\mathbf{y}$ is an $M$-dimensional dense vector containing a condensed representation of $\mathbf{x} $, where:
			\begin{equation}
				K < M << N 	
			\end{equation}

			It can then be proven that using \emph{random projection} i.e. using a random dimensionality reduction matrix $\Phi$, will allow for recovery of $\mathbf{x}$ with little to no information loss, with overwhelming probability.
		
			% sub-subsection condensed_representation (end)

		\subsubsection{Basis} % (fold)
		\label{sub:basis}
			
			When a signal is sparse in a domain different from the sampled domain, one can make a transformation, by way of a transformation matrix.

			\begin{equation}
				\mathbf{y} = \Phi \Psi \mathbf{x}
				\label{eq:transMat}	
			\end{equation}

			In (\ref{eq:transMat})) $\Psi$ is an $N \times N$ transformation matrix e.g. the Discrete Fourier Transform (DFT) or Discrete cosine Transform (DCT).
			The dimensionality reduction matrix and the transformation matrix can be combined to an augmented dimensionality matrix.

			\begin{equation}
				\mathbf{y} = \Phi \Psi \mathbf{x} = \Phi' \mathbf{x}
				\label{eq:augDimReduc}	
			\end{equation}

			This, combined with the fact that random projection is an option negates the need for a transformation matrix, as a random projection of any matrix is just another random matrix.
			It is then only necessary to be aware of the domain in which the signal is sparse when one is reconstructing the signal, not when sampling or transmitting.
			One needs only to be able to know or construct the random projection matrix used to compress the signal.
			This can be done by using pseudo-random generator with a know or predictable seed.

			% sub-subsection basis (end)

		\subsection{Signal Reconstruction} % (fold)
		\label{sub:signal_reconstruction}

			\fxwarning{WRITE THIS SECTION} 
		
			% subsection signal_reconstruction (end)


	
		% subsection sparsity_and_basis (end)


	% \subsection{Sampling Matrix} % (fold)
	% \label{sub:sampling_matrix}

	% 	% subsection sampling_matrix (end)


	\subsection{Random Access Compressive sensing} % (fold)
	\label{sub:random_access_compressive_sensing}

		In this section a brief introduction to the ideas behind RACS.
		\footnote{For more information see the original article \cite{Fazel2011}}.
		This will be done in two parts, one about the random access transmission protocol and one about the proposed method of CS ind the spatial domain.

		Lastly some methods for enhancing the performance of the CS method are proposed.
		
		\subsection{Compressive Sensing in Spatial Domain} % (fold)
		\label{sub:compressive_sensing_in_spatial_domain}

			The foundation of the proposed method for CS in the spatial domain is a regular grid of $I \times J = N$ wireless sensors spanning an area $A = Nd^2$, where $d$ is the distance between sensors in orthogonal directions.
			Each sensor knows its own position in the grid and can sample a physical phenomenon.
			The sensors are able to transmit their data to a computationally powerful sink called a \emph{Fusion Center} (FC).

			The data at a certain sensor within time frame $n$ is represented as
			\begin{equation}
				u_ij(n) = u(x_i, y_j, n)
			\end{equation}
			%
			where $(x_i, y_j)$ is the sensors location in the grid.

			The frame duration is dependent on the coherence time $T_coh$ of the signal i.e. for how long can one assume that the data has not changed significantly.
			The map of the data of all the sensors can then be represented as
			%
			\begin{equation}
				\mathbf{U}(n) =
			 	\begin{bmatrix}
			  		u_{11}	&	\cdots 	& 	u_{1J}	\\
			  		\vdots	&	\vdots	&	\vdots	\\
					\cdots 	& 	u_{IJ}	& 	\cdots 	\\
					\vdots	&	\vdots	&	\vdots	\\
					u_{I1} 	& 	\cdots 	& 	u_{IJ}
				\end{bmatrix}
			\end{equation}
			%
			$\mathbf{U}(n)$ is an $I \times J$ dense matrix, with entries that vary slowly in space.
			Because of the slow spatial variation of the data, the corresponding Fourier coefficients, $\mathbf{V}(n)$, are presumed sparse.
			Further, \cite{Fazel2011} states that $\mathbf{v}(n) = \Psi_{DFT}\ \mathbf{u}(n)$, where $\mathbf{v}(n) = \text{vec}(\mathbf{V}(n))$ is the $N \times 1$ vector gathered by stacking the columns of $\mathbf{V}(n)$.

			Data from $M$ sensors is then randomly selected for use in reconstruction.
			The selected data $y(n)$ is represented as
			%
			\begin{equation}
				\mathbf{y}(n) = 
					\mathbf{R}(n) \mathbf{u}(n) + \mathbf{z}(n)
					= \mathbf{R}(n) \Psi_{IDFT}\ \mathbf{v}(n) + \mathbf{z}(n)
			\end{equation}
			%
			where $\mathbf{R}(n)$ is an $M \times N$ random selection matrix consisting of $M$ rows of $I^N$ selected uniformly at random.

			The reconstruction problem is then:
			%
			\begin{equation}
				\min_{\mathbf{\hat{v}(n)}} 
					\|\mathbf{\hat{v}(n)}\|_{l_1} 
					\ \ \text{subject to} \ \ 
					\mathbf{R}(n) \Psi_{IDFT}\ \mathbf{\hat{v}}(n) = \mathbf{y}(n)
			\end{equation}
			%
			The minimum number of samples needed for sufficient sensing $N_s$ is stated in theory to be
			%
			\begin{equation}
				N_s = CS \log{N}
				\label{eq:NsTheory} 
			\end{equation}
			%
			where $S$ is the sparsity of the signal and $C$ is a constant, independent of $N$ and $S$.
			The authors of RACS refer to \cite{Candes2006} for a upper bound on C.
			Another method is to determine $N_s$ experimentally.
			This method often produces even lower numbers for $N_s$ than (\ref{eq:NsTheory})).
		
			% subsection compressive_sensing_in_spatial_domain (end)


		\subsection{Random Access} % (fold)
		\label{sub:random_access}

			The proposed RACS scheme is as follows:

			\begin{enumerate}[label=\bfseries Step \arabic*:]
				\item 
					At the beginning of each frame a sensor node $i$ randomly chooses to participate in sensing with probability $p$, or stay inactive.
				\item 
					If node $i$ is selected for sensing, it samples the quantity of interest and transmits it along with its location.
				\item
					Node $i$ selects a transmission-delay $\theta_{i}$ before attempting to send its packet.
				\item
					The FC collects singleton packets, and discards any colliding packets.
				\item
					At the end of the frame the FC reconstructs  the data from the received measurements.

			\end{enumerate}

			To figure out which probability of sensing $p$ will result in enough collision-free packets arriving necessary for reconstruction, one first has to make a model of the transmission.

			The provability of a packet arriving without collision is
			\begin{equation}
				q = pe^{-2\frac{NpT_p}{T-T_p}}
			\end{equation}
			%
			where $T$ is the frame length and $T_p$ is the transmission time for one packet.

			The number of collision-free packets received in a frame, $K$ i a random variable with a binomial distribution:
			\begin{equation}
				P_K(k) = \text{Prob}\{K=k\} = B(N,q) =
					\binom{N}{k} q^k(1-q)^{N-k}
			\end{equation}
			%
			In order to guarantee with a \emph{probability of sufficient sensing} of $P_s$ that $K >= N_s$ one needs to look at the complementary cumulative distribution function of $K$, $Q_K(N_S)$.
			The condition can be expressed as:
			\begin{equation}
				\text{Prob}\{K \geq N_s\} = Q_K(N_S) = (1-P_K(N_s)) \geq P_s
			\end{equation}

		

			% subsection random_access (end)


		\subsection{Suggestions for improving the RACS method} % (fold)
		\label{sub:suggestions_for_improving_the_racs_method}
		

			% subsection suggestions_for_improving_the_racs_method (end)

		% subsection random_access_compressive_sensing (end)

	% section theory (end)

\end{document}
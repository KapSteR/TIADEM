%!TEX root = Main.tex
\documentclass[Main]{subfiles}

\begin{document}
	\section{Conclusion} % (fold)
	\label{sec:conclusion}

		In this project a simulation of RACS in MATLAB was successfully implemented.
		The results presented in this report confirms the theory put forth by \cite{Fazel2011}.
		Overall the results point to RACS being very beneficial for wireless sensor networks in general, not just underwater networks.
		The savings in energy consumption from not just compressing data, but deliberately omitting sensing and transmitting, unless selected are significant, with very limited loss in fidelity of the data.
		
		Also, the proposed random access protocol completely frees the nodes from listening, further saving energy.
		In general, the decentralized nature of the protocol makes efficient implementation easy, with little computational overhead.

		The RACS method seems to have issues with harmonic overtones in the signal vector stemming from discontinuities introduced by stacking the columns of the data map.
		In this report suggestions for improving the RACS method was presented.
		Especially, a suggestion for changing the way the data map was indexed has shown promise.
		The \emph{snake} method has in this project shown itself to be able to reduce the compression ratio by \emph{one third} compared to regular RACS.
		
		To accurately state the potential of this method, further analysis and test is necessary.
		The indication is however that it has the potential to very significantly improve the compression performance of RACS, and by extension network energy consumption and hence network lifetime, in a power constrained network.

				
		% section conclusion (end)

\end{document}
%!TEX root = Main.tex
\documentclass[Main]{subfiles}

\begin{document}

\section{Data} % (fold)
\label{sec:data}
	
	This section introduces the data used in this project.

	The dataset, called \emph{Global 1-km Sea Surface Temperature} (G1SST), is as the name implies a 1 day global snapshot of sea surface temperatures in a regular grid, with data points spaced $1\ km$ apart.

	The data is provided by \emph{NASA Jet Propulsion Laboratory (JPL)} \cite{G1SST:Online}.
	It is collected from a number of sources including satellites, ships and buoys and is combined to a seamless data map by JPL.
	The data spans most of the globe: $\pm 80\degree \ latitude,\ 0- 360\degree \ longtitude$.

	Each day a new dataset is created, and is available on-line in a \emph{.nc}-formatted database.
	Apart from raw data, information about precise global coordinates of datapoints, time of creation and and masks of sea/land/ice etc. are among others also provided.

	This dataset was chosen in two criteria: (I) It is high resolution spatial data (II) It varies slowly in space, indicating a level of sparsity in some domain.

	In this project only data from one day, September 15$^{th}$ 2014, is used. An example of this data is shown below in figure \ref{fig:ExampleSSTData}.
	
	\begin{figure}[H]
		\centering 
		\includegraphics[width=\textwidth]{ExampleSSTData.eps}
		\caption{Example of SST data taken from a patch of the North Sea}
		\label{fig:ExampleSSTData}
	\end{figure}

	% section data (end)

\end{document}
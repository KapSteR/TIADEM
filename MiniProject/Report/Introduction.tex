%!TEX root = Main.tex
\documentclass[Main]{subfiles}

\begin{document}

\section{Introduction} % (fold)
\label{sec:introduction}
	In this report \emph{Random Access Compressed Sensing (RACS)}, as proposed by \cite{Fazel2011} is attempted implemented in MATLAB, with the aim of evaluating its use with natural signals, other than those presented in in the article.
	Here data about sea surface temperature is tested.

	This report has been produced as the result of a `Mini-project' in the course \emph{Advanced Embedded Sensor Networks (TIADEM)} at \emph{Aarhus University, Department of Engineering}.
	The goal of the project is to gain further understanding of the theory of \emph{Compressed Sensing (CS)}, by implementing it in MATLAB and applying it to real world data.
	The data in this project is gathered from a mixture of satellites, ships and buoys.
	This data stands in for collected data from an actual sensor network.
	The network communication is then modeled and simulated.

	The scope of this project is to simulate the random access protocol proposed by \cite{Fazel2011} and apply CS theory, to show how well (or not) the presented theories and models hold up.
	Some suggestions on how to improve these will also be presented.
	A real world implementation of the protocol will not be considered, as it is beyond the scope of this report.

	% section introduction (end)


\end{document}
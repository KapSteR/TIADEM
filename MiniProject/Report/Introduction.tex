%!TEX root = Main.tex
\documentclass[Main]{subfiles}

\begin{document}

\section{Introduction} % (fold)
\label{sec:introduction}
	In this report \emph{Random Access Compressed Sensing (RACS)}, as proposed by \fxwarning{Insert ref to paper} is attempted implemented in MATLAB, with the aim of evaluating its use with other natural signals than those presented in \fxwarning{See above}.
	Here data about sea surface temperature, gathered from satellites is used.

	This report has been produced as the result of a `Mini-project' in the course \emph{Advanced Embedded Sensor Networks (TIADEM)} at \emph{Aarhus University, Department of Engineering}.
	The goal of the project is to gain further understanding of the theory of \emph{Compressed Sensing (CS)}, by implementing it in MATLAB and applying it to real world data.

	The scope of this project is to simulate the random access protocol proposed by \fxwarning{ref!!} and apply CS theory, to show how well (or not) the presented theories and models hold up.
	Some attention will also be given to how different specification parameters of a RACS system affect its performance. \fxnote{check if this holds} 
	A real world implementation of the protocol will not be considered, as it is beyond the scope of this report.



% section introduction (end)


\end{document}
%!TEX root = Main.tex
\documentclass[Main]{subfiles}

\begin{document}

\section{Design And Implementation} % (fold)
\label{sec:design_and_implementation}

	\subsection{Data Input} % (fold)
	\label{sub:data_input}

		The MATLAB script \emph{InputData.m} uses build in functions to extract data from the downloaded G1SST database.
		It takes as input an upper and lower bound and an eastern and western bound in between which it is supposed to a square set of data.

		The data is converted from Kelvin to Celsius, to make it more semantically understandable.
		The data is then saved to a \emph{.mat}-file for use in other scripts.

		Finally the data is shown on a colored contour-plot (see Figure \ref{fig:ExampleSSTData}), for illustration and inspection purposes.

	
		% subsection data_input (end)

	\subsection{Sparsity Estimation} % (fold)
	\label{sub:sparsity_estimation}

		The MATLAB script \emph{EvalSparsity.m} takes as input to \emph{.mat}-file described in Section \ref{sub:data_input}.

		It vectorizes the data map by stacking either the rows or the columns.
		It then evaluates the sparsity by taking either the DFT or the DCT of the signal.
		The sparsity is the number, $k$, of coefficients greater than 10 or 0.1, respectively.

		It then tries to do a reconstruction using $M = 6*k$ randomly selected samples, and calculates the normalized error:
		%
		\begin{equation}
			e_{normalized} = \frac
				{\|\mathbf{\hat{u}}(n) - \mathbf{u}(n)\|_{l2}}
				{\|\mathbf{u}(n)\|_{l2}}
		\end{equation}




	
		% subsection sparsity_estimation (end)

	\subsection{Estimaton of $q_s$ and $p_s$} % (fold)
	\label{sub:estimaton_of_q_s_and_p_s_}
	
		% subsection estimaton_of_q_s_and_p_s_ (end)

	\subsection{Simulation of Random Access Transmission} % (fold)
	\label{sub:simulation_of_random_access_transmission}



		\fxnote{Mean K vs p figure}
		\fxnote{reconstruction error vs p figure}
	
		% subsection simulation_of_random_access_transmission (end)

	\subsection{Data Reconstruction (Sensor Fusion)} % (fold)
	\label{sub:data_reconstruction}
	
		% subsection data_reconstruction (end)

	
	\subsection{Comparison of Suggested Methods} % (fold)
	\label{sub:comparison_of_suggested_methods}

		\fxnote{comparison of signal and DCT with and without snake}
	
		% subsection comparison_of_suggested_methods (end)

	% section design_and_implementation (end)


\end{document}